
Reverse osmosis is one of the most common water purification techniques and is used in applications ranging from salt water desalination plants to medical devices. Baxter company develops medical devices and water purification devices used in medical applications using reverse osmosis technique. \\
\\
In this thesis a different flow path is investigated. By replacing a single feed water pump with two independent pumps investigations of the possibilities of optimising the performance of the membrane is done. The goal is to investigate any advantages with the setup and to design a control application that is able to optimise the performance of the system for all operating conditions within the operating range. The optimisation is isolated to the performance of the reverse osmosis membrane used in the water device by Baxter.\\
\\
Theory of the reverse osmosis identifies two important parameters that can improve the membrane performance, pressure and flow. Two parameters that affect the purity of the product water delivered by the water device is conductivity and temperature, which has a high negative impact of the membrane performance.\\
\\
The second pump offers another degree of freedom which allows the pressurisation and flow over the membrane to be independently controlled. \\
\\
The method for the investigations contain both modelling in Matlab tool Simscape and real test series on a physical rig built at Baxter. Tests on current setup used by Baxter today and the two pump system is performed. \\
\\
Result shows on some improvements of the membrane performance by using the two pump solution. Especially in critical working areas as high temperature or high conductivity, or a combination of both.\\ 
\\
 