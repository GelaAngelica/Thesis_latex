% !TEX root = /Users/Gela/Desktop/Thesis_latex/thesis.tex
\section{Semi-permeable membrane}
A membrane is defined as a barrier between two homogeneous phases. The process is a continuos steady-state operation consisting three streams: feed, permeate and reject. Main concern in the process boundary is the semipermeable barrier that selectively allows the passage of some components but not others. \cite{Singh}

\section{Osmosis}
The osmosis process occurs when two solutions of different chemical concentration are separated by a semi-permeable membrane. The two different solutions will try to reach equilibrium. The solution with less concentration will have a natural tendency to migrate through the membrane over to the side with higher concentration.  
Osmosis is a naturally occurring penomenon and one of the most important processes in nature. The pressure that occurs is called the osmotic pressure. The phenomenon can be seen in PICTURE. 

\section{Reverse osmosis}
The reverse osmosis(RO) process is the reverse process of the osmosis. When pressure is applied to a semipermeable membrane, the water molecules are forced through the semipermeable membrane and the contaminants are not allowed true. The amount of pressure required depends on the salt concentration of the water. In order to gain reverse osmosis the pressure applied must be greater than the osmosis pressure. The membrane employs cross filtration rather than standard filtration. With cross filtration, the solution passes through the filter with two outlets. One solution passes true the membrane and is called permeate and is the filtered solution. The other solution can be drained or be fed back into the filtering system. The contaminants build up att the surface area and it is of great importance to try to sweep them away and hold the surface clean. If the contaminants builds up the performance of the membrane will decrease, and cleaning with chemicals or heat water might be necessary.




\section{Modeling}
%Teorin bakom modelleringen
\section{System identification}
%H�r ska teorin om systemidentification

\section{Control theory}

%All reglerteori




