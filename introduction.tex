% !TEX root = /Users/Gela/Desktop/Thesis_latex/thesis.tex

\section*{Background}

The Water Technologies department at Baxter develops water systems for use in mixing fluid for dialysis treatments. The water quality is important to not create any harm to the patients when using the final product. The water systems used for water purification are using the reverse osmosis (RO) method as the finest level of filtration. It remove impurities, as salt and inorganic molecules from the water\cite{Dow}.\\

In a RO-system the feed water is pressurized by a pump and forced through the RO-membrane to overcome the osmotic pressure. The RO-membrane is a semi-permeable membrane and let water passes freely true the membrane creating a purified product stream. \\

The pump in the current system has two purposes, creating a pressure to overcome the osmotic pressure and creating a flow on the reject side of the RO-membrane to prevent aggregation of impurities on the membrane surface.\\



\section*{Motivation}
By using two pumps instead of one in the RO-system it will be possible to control the pressure on the module and the flow on the reject side independently and thus get better possibility to optimize the performance of the RO-system, focusing on reducing impurities and water consumption. \\

As the current model does not take temperature dependencies in concern, the model will be redesigned in order to handle temperature dependencies.  

\section*{Goal}
The purpose of this masters thesis is to evaluate the feasibility of replacing the main RO-pump with two pumps, one for controlling the flow through the membrane and one for controlling the pressure. \\

To achieve good performance it will be necessary to design a realistic model of the system, once the model has been designed and tested a control algorithm is to be developed. This algorithm, should be able to control the flow and pressure over the RO-membrane to maximize the efficiency of the filter while minimizing the amount of waste water that is produced. \\

The temperature dependencies will be taken in concern in the new model.


\section*{Method}

In order to investigate the performance of the current system and to compare it with the new model following steps will be evaluated:

\begin{itemize}
\renewcommand\labelitemi{-}
    \item Research on the RO-membrane that is implemented in the system  
    \item Research on previous work on the field
    \item Modelling of the system to identify suitable component properties and design of the flow path
    \item Design of control algorithms
    \item Control simulations
    \item Implementation in a test rig to verify the performance of the system
    \item Run tests to determine the performance
    \item Improve if possible
\end{itemize}




