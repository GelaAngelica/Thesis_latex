% !TEX root = /Users/Gela/Desktop/Thesis_latex/thesis.tex
\section{Modeling}
A physical model of the membrane were made and the given results can be seen below. In figure \ref{fig:simscape} the flowchart is given. The RO-membrane is simulated with three nodes; feed, product and reject, and an extra input temp. The temp node gives freedom to simulate the behaviour of the membrane in different temperature ranges. The model consists the pump, pipes, valves and tank. The water in the tank can be set to contain any salt koncentration to be able to simulate different conductivity values. The speed of the pumps can be changed to change the flow and pressure characteristic over the membrane in the model.\\
\\
\begin{figure}[h]
\label{fig:simscape}
\centering
\includegraphics[width=\textwidth]{simscape_fc1.PNG}
\caption{Model made in Matlab tool Simscape}
\end{figure}


\subsection{Temperature}
In figure \ref{fig:temp} the simulated temperature, 278-316 K is shown. All plots \ref{fig:msaltf} - \ref{fig:conden} shows the behaviour over a simulated time of 2000 s and a temperature range of 278-316 K. Pump speed is kept constant. The temperature correction factor, TCF, in figure \ref{fig:tcf} is the temperature dependent parameter implemented in the simulated model to receive the differences of the behaviour of the membrane. At 298 K TCF is equal to 1. Below and above it is adjusted to compensate for the differences of the membrane behaviour. \\
\\
\begin{figure}[h]
  \centering
  \includegraphics[width=0.5\linewidth]{temp.PNG}
  \caption{Temperature}
  \label{fig:temp}
\end{figure}
\begin{figure}[h]
\centering
    \centering
    \includegraphics[width=0.5\textwidth]{tcf.PNG}
    \caption{Temperature correction factor}
    \label{fig:tcf}
\end{figure}

\subsection{Salt koncentration}

In figure \ref{fig:msaltf} the salt koncentration in kg/s on feed side of the membrane is shown. The concentration increases over the simulated time and change in temperature. \\
In figure \ref{fig:msaltp} the salt koncentration on product side is shown. Due to the mass balance equation in XXXX the sign is negative, and the koncentration increases with temperature.\\
In figure \ref{fig:msaltr} the salt koncentration on reject side is shown. It increases a little with temperature (negative sign due to the mass balance equations).\\
The product salt koncentration increases with from a value of  0.8 - 0.96 kg/s. The product water koncentration increases from (-) 0.2 - 1.8 kg/s. The reject salt koncentration increases from (-) 0.8 - 0.96 kg/s.\\
\\
\begin{figure}[h]
\centering
    \includegraphics[width=0.5\textwidth]{msalt_feed.PNG}
    \caption{Salt concentration feedwater}
    \label{fig:msaltf}
\end{figure}
\begin{figure}[h]
\centering
  \includegraphics[width=0.5\linewidth]{msalt_prod.PNG}
  \caption{Salt concentration productwater}
  \label{fig:msaltp}
\end{figure}
\begin{figure}[h]
  \centering
  \includegraphics[width=0.5\linewidth]{msalt_rej.PNG}
  \caption{Salt concentration reject water}
  \label{fig:msaltr}
\end{figure}
Figure \ref{fig:qf} - \ref{fig:qrej} shows the flow in the three nodes; feed, product and reject. The mass balance equation in XXXX gives negative sign on reject and product side. The feed water flow increases negligible from 8.441-8.444 l/min. Product water flow increases from (-) 0.62-1.42 l/min. Reject water flow decreases from (-) 7.82-7.2 l/min. \\
\\

\subsection{Flow}
\begin{figure}[h]
  \centering
  \includegraphics[width=0.5\linewidth]{q_feed.PNG}
  \caption{Flow feed side}
  \label{fig:qf}
\end{figure}
\begin{figure}[h]
  \centering
  \includegraphics[width=0.5\linewidth]{q_prod.PNG}
  \caption{Flow product side}
  \label{fig:qp}
\end{figure}
\begin{figure}[h]
  \centering
  \includegraphics[width=0.5\linewidth]{q_rej.PNG}
  \caption{Flow reject side}
  \label{fig:qrej}
\end{figure}

\subsection{Pressure}
Figure \ref{fig:deltap} shows pressure difference from feed side to product side. The pressure decreases from (-) 11.5-7.7 bar when temperature changes from 278-316 K. \\
\\
\begin{figure}[h]
  \centering
  \includegraphics[width=0.5\linewidth]{deltap.PNG}
  \caption{Pressure drop feed side to product side}
  \label{fig:deltap}
\end{figure}

\subsection{Conductivity}
Figure \ref{fig:conden} displays the conductivity in feed, product and reject side. The conductivity in all nodes increases when temperature increases.\\
\\
\begin{figure}[h]
  \label{fig:conden}
  \includegraphics[width=1.1\linewidth]{cond.PNG}
  \caption{Conductivity in feed, reject and product side}
\end{figure}

\newpage


\section{Flowchart investigation}
By changing the flow path in the test setup both the one pump system and two pump system could be investigated and their performance could be compared. Both Systems were considered fulfilling most of the requirements, section \ref{framing}, for an updated version. \\

One pump system, figure \ref{fig:FlowCInves1}, The one pump system was designed to use both a tank and the recirculation loop as a water source and to create pressure by generating a large flow over the membrane and recirculation restrictor. \\


Two pump system, figure \ref{fig:Sys2}, In the two pump system the water path was modified so that the feed pump only used a tank as a source and pressureized the entire recirculation loop. The recirculation pump was used to recirculate the already\\ pressureized water within the recirculation loop.\\


An overview of the systems can be seen below.\\
\begin{figure}[h]
\centering
\begin{minipage}{.5\textwidth}
    \centering
    \includegraphics[width=0.8\textwidth]{Sys1}
    \caption{One pump system}
    \label{fig:System1}
\end{minipage}%
\begin{minipage}{.5\textwidth}
  \centering
  \includegraphics[width=.8\linewidth]{Sys2}
  \caption{Two pump system}
  \label{fig:System2}
\end{minipage}
\end{figure}

\newpage

\section{Implementation Test Rig}


\subsection{Connections}
In Figure (\ref{fig:PressConn}-\ref{fig:PumpConn}) all connections in the test rig is displayed. The whole rig can be seen in \ref{fig:Rig1} and \ref{fig:Rig2}

\begin{figure}[h]
    \centering
    \includegraphics[width=0.5\textwidth]{Rig1}
    \caption{The rig built at Baxter Lund AB, with RO-membrane, pumps, pipes, flowmeter, measuremet sensors and valves}
    \label{fig:Rig1}
\end{figure}

\begin{figure}[h]
    \centering
    \includegraphics[width=0.5\textwidth]{Rig2}
    \caption{The full setup built at Baxter Lund AB, with simulink implementation, GUI, display and water bath}
    \label{fig:Rig2}
\end{figure}


\begin{figure}[h]
    \centering
    \includegraphics[width=0.6\textwidth]{PressConn}
    \caption{Connections Pressure sensors}
    \label{fig:PressConn}
\end{figure}

\begin{figure}[h]
    \centering
    \includegraphics[width=0.6\textwidth]{C3Conn}
    \caption{Connections measurement blocks, C3}
    \label{fig:C3Conn}
\end{figure}

\begin{figure}[h]
    \centering
    \includegraphics[width=0.6\textwidth]{ValveConn}
    \caption{Connections Drain Valve}
    \label{fig:ValveConn}
\end{figure}

\begin{figure}[h]
    \centering
    \includegraphics[width=0.6\textwidth]{PumpConn}
    \caption{Connections pumps}
    \label{fig:PumpConn}
\end{figure}
\begin{figure}[h]
    \centering
    \includegraphics[width=0.6\textwidth]{Display}
    \caption{The Display with all key values, read from sensors in the rig}
    \label{fig:display}
\end{figure}
\begin{figure}[h]
    \centering
    \includegraphics[width=0.6\textwidth]{GUI}
    \caption{The GUI implemented in Simulink and used to control the rig}
    \label{fig:gui}
\end{figure}

\newpage



\section{Investigation on membrane behaviour}

In order to compare the two systems and understand how the membrane performed in different working conditions both systems needed to be tested. The tests were conducted by controlling the temperature, pumps and the conductivity in the recirculation loop and log how the different conditions affected the system and the membrane. 

The tests were conducted by changing the temperature, recirculation conductivity and pump speed and meassure how the system behaved once it had reached steady state. The tests were divided into three test sequences. One test sequence was performed with room temperatured water (19 C), one with water heated to 30 C and in the last test sequence the water was heated to 40 C. Every test sequence included data from 8 steady state points with different settings on feed conductivity and pump speed. The test sequences are displayed in the table below. 

\begin{table}[h]
\centering
\begin{tabular}{|p{1.4cm}||p{2cm}|p{3.2cm}|p{1.8cm}|}
 \hline
 \textbf{Steady state }&Temperature&Feed Conductivity&Motor effect \\
 \hline
 1.1 & 18 $^\circ$C   & 280 \SI{}{\micro\siemens} & 60 \% \\
 1.2   &  18 $^\circ$C   & 500 \SI{}{\micro\siemens} & 60 \% \\
 1.3 &  18 $^\circ$C  &1000 \SI{}{\micro\siemens} & 60 \% \\
 1.4 &  18 $^\circ$C  &1000 \SI{}{\micro\siemens} & \textbf{80 \%} \\
 1.5 &18 $^\circ$C &2000 \SI{}{\micro\siemens}& 60 \%\\
 1.6 &18 $^\circ$C  &2000 \SI{}{\micro\siemens}& \textbf{80 \%}\\
 1.7   &18 $^\circ$C & 3000 \SI{}{\micro\siemens}&60 \% \\
 1.8   &18 $^\circ$C&3000 \SI{}{\micro\siemens}& \textbf{80 \%}\\
 \hline
 2.1 & 30 $^\circ$C & 280 \SI{}{\micro\siemens}&60 \%\\
 2.2 & 30 $^\circ$C &500 \SI{}{\micro\siemens}& 60 \%\\
 2.3 & 30 $^\circ$C&1000 \SI{}{\micro\siemens}& 60 \%\\
 2.4 & 30 $^\circ$C&1000 \SI{}{\micro\siemens}& \textbf{80 \%}\\
 2.5 & 30 $^\circ$C&2000 \SI{}{\micro\siemens}& 60 \%\\
 2.6 & 30 $^\circ$C&2000 \SI{}{\micro\siemens}& \textbf{80 \%}\\
 2.7 & 30 $^\circ$C& 3000 \SI{}{\micro\siemens}&60 \%\\
 2.8 & 30 $^\circ$C& 3000 \SI{}{\micro\siemens}&\textbf{80 \%}\\
 \hline 
 3.1 & 40 $^\circ$C& 280 \SI{}{\micro\siemens}& 60 \%\\
 3.2 & 40 $^\circ$C &500 \SI{}{\micro\siemens}& 60 \%\\
 3.3 & 40 $^\circ$C  & 1000 \SI{}{\micro\siemens}& 60 \%\\
 3.4 & 40 $^\circ$C  & 1000 \SI{}{\micro\siemens}& \textbf{80 \%}\\
 3.5 & 40 $^\circ$C&2000 \SI{}{\micro\siemens}& 60 \%\\
 3.6 & 40 $^\circ$C &2000 \SI{}{\micro\siemens}& \textbf{80 \%}\\
 3.7 & 40$^\circ$C &3000 \SI{}{\micro\siemens}& 60 \%\\
 3.8 & 40$^\circ$C &3000 \SI{}{\micro\siemens}& \textbf{80 \%}\\
\hline
\end{tabular}
\caption{Testcases}
    \label{tab:test cases} 
\end{table}


\subsection{Current system, Test sequence 1, part 1}

The water in the tank was heated while the test was running. Because of this, the test was split up in two parts, first the motor was set to 60\% and steady state 1.1, 1.2, 1.3, 1.5 and 1.7 were investigated. In the part 2, the motor was set to 80 \% and steady state 1.4, 1.6 and 1.8 were investigated. 
\begin{figure}[H]
    \centering
    \includegraphics[width=1.1\textwidth]{overview20_60}
    \caption{Test 1, Current system, 18 degrees celsius. Steady states 1.1, 1.2, 1.3, 1.5 and 1.7 }
    \label{fig:PressConn}
\end{figure}

\newpage

\subsection{Current system, Test sequence 1, part 2}
  
\begin{figure}[H]
    \centering
    \includegraphics[width=1.1\textwidth]{overview20_80}
    \caption{Test 1, Current system, 18 degrees celsius. Steady states 1.4, 1.6 and 1.8}
    \label{fig:PressConn}
\end{figure}

\newpage

By post-proccesing the data from test one in Matlab it was possible to visually show how the system parameters were affected by the changed pump speed and feed conductivity. 


insert table, results on how the different graphs changed!!!

\begin{figure}[H]
    \centering
    \includegraphics[width=1.1\textwidth]{Key20}
    \caption{Caption missing}
    \label{fig:PressConn}
\end{figure}

\newpage

\subsection{Current system, Test sequence 2}

The second test was carried out by setting the heater bath to 30 degrees celsius and and adjusting the conductivity and pump speed according to the test plan. Since the water was much warmer than the air in the room, the heating caused by the pump was not as prominent and allowed all steady states to be examined in one continous test.

\begin{figure}[H]
    \centering
    \includegraphics[width=1.1\textwidth]{overview30}
    \caption{Test 2, Current system, 30 degrees celsius. Steady states 1.1, 1.2, 1.3, 1.4 1.5, 1.6, 1.7 and 1.8}
    \label{fig:PressConn}
\end{figure}

\newpage


The data from the test was post processed in Matlab in exactly the same way as the previous test.

\begin{figure}[H]
    \centering
    \includegraphics[width=1.1\textwidth]{Key30}
    \caption{Caption missing}
    \label{fig:PressConn}
\end{figure}

\newpage

\subsection{Current system, Test sequence 3}

Finally the heating bath was set to 40 C and the test sequence was performed just like test sequence 2. 

\begin{figure}[H]
    \centering
    \includegraphics[width=1.1\textwidth]{overview40}
    \caption{Caption missing}
    \label{fig:PressConn}
\end{figure}

\newpage

Post processing in matlab generated the following data from the steady states.

\begin{figure}[H]
    \centering
    \includegraphics[width=1.1\textwidth]{Key40}
    \caption{Caption missing}
    \label{fig:PressConn}
\end{figure}

\newpage

In order to understand how the system current performed in different working conditions all plots from the post processing in Matlab was put togheter. 

\subsection{Net driving pressure}

Net driving pressure was decreased when the temperature was increased. Higher feed conductivity resulted in a decreased net driving pressure. As expected, running the feed pump at a higher RPM also increased the net driving pressure.

\begin{figure}[H]
    \centering
    \includegraphics[width=0.8\textwidth]{NDP}
    \caption{Net driving pressure}
    \label{fig:PressConn}
\end{figure}

\subsection{Permeate flow}

According to theory, Net driving pressure has a direct effect on permeate flow. When the feed pump was increased more water was pushed through the membrane. Increased water temperatures caused a much higher permeate flow. For instance, the permeate flow increased by around 50\% when the temperature was increased from 19 C to 40 C and the pump was running at 60\%. Due to the increased osmotic pressure, permeate flow decreased when the feed conductivity increased

\begin{figure}[H]
    \centering
    \includegraphics[width=0.8\textwidth]{permFlowCurrent}
    \caption{Caption missing}
    \label{fig:PressConn}
\end{figure}

\subsection{Recovery}

Warmer water enabled more feed water to pass through the membrane and therefore the recovery was increased. Increased conductivity reduced recovery due to the increased osmotic pressure.

\begin{figure}[H]
    \centering
    \includegraphics[width=0.8\textwidth]{Recovery}
    \caption{Caption missing}
    \label{fig:PressConn}
\end{figure}

\subsection{Salt rejection}

By looking at plot 5.13 the deterimental effects of both increased temperature and feed conductivity can be seen. The negative effect of increased feed conductivity was much more prominent at 40 C than 30C which means that the performance of the membrane decreased exponentially with higher temperature. Increased feed pump pressure resultet in better salt rejection and the positive effect of increased pump pressure was larger when the system was hot. Temperature was the parameter that decreased salt rejetion the most and by comparing how the system performed when the pump and feed conductivity was set to 60\% and 3000uS/cm at 19C and 40C it can be seen that the salt rejection decreased from 98.5\% to 95.5 \%. From the experiment it can also be concluded that the system perform much better at low temperature and feed conductivity that at high temperature and feed conductivity. 

\begin{figure}[H]
    \centering
    \includegraphics[width=0.8\textwidth]{SaltRejection}
    \caption{Saltrejection}
    \label{fig:PressConn}
\end{figure}
 
\subsection{Permeate conductivity}

Permeate conductivity was directly proportional to salt rejection at a certain termperature and feed conductivity. The black line in plot 5.14 show the critical permeate conductivity that the system should be able to maintain and from the plot it is possible to se how high the conductivity can be in the recirculation loop without exceeding this limit. The operational area for the different temperature and pump pressure can be seen in plot TBD below.

\begin{figure}[H]
    \centering
    \includegraphics[width=0.8\textwidth]{PermCond}
    \caption{Conductivity}
    \label{fig:PressConn}
\end{figure}

!!INSERT TABLE!!

\subsection{Water efficiency}

Water efficiency increased when the temperature increased due to more permeate water being generated by the same feed pressure.

\begin{figure}[H]
    \centering
    \includegraphics[width=0.8\textwidth]{Efficiency}
    \caption{Efficiency}
    \label{fig:PressConn}
\end{figure}

\subsection{System 2}

Test: increased feed pressure:

Add all plots from test with two pump system. TBD 

Section not finished! 



Test: increased feed pressure


Sen körde vi tester för att hitta en optimal kurva.

Tanken är att presentera resultat, beskriva varför vi valde att optimera systemet på det här sättet och till slut visa hur vi testade oss fram till och beräknade en kurva för att omvända temperatur till en setpoint för permeatflöded.

\begin{figure}[H]
    \centering
    \includegraphics[width=1.1\textwidth]{FeedPumpIncrease21}
    \caption{Insert caption}
    \label{fig:PressConn}
\end{figure}


\begin{figure}[H]
    \centering
    \includegraphics[width=1.1\textwidth]{FeedPumpIncrease21Key}
    \caption{Caption missing}
    \label{fig:PressConn}
\end{figure}

30


\begin{figure}[H]
    \centering
    \includegraphics[width=1.1\textwidth]{FeedPumpIncrease30}
    \caption{Caption missing}
    \label{fig:PressConn}
\end{figure}


\begin{figure}[H]
    \centering
    \includegraphics[width=1.1\textwidth]{FeedPumpIncrease30Key}
    \caption{Caption missing}
    \label{fig:PressConn}
\end{figure}

40


\begin{figure}[H]
    \centering
    \includegraphics[width=1.1\textwidth]{FeedPumpIncrease40}
    \caption{Caption missing}
    \label{fig:PressConn}
\end{figure}


\begin{figure}[H]
    \centering
    \includegraphics[width=1.1\textwidth]{FeedPumpIncrease40Key}
    \caption{Caption missing}
    \label{fig:PressConn}
\end{figure}

Recovery Increase

\begin{figure}[H]
    \centering
    \includegraphics[width=1.1\textwidth]{RecIncrease40}
    \caption{Caption missing}
    \label{fig:PressConn}
\end{figure}

\begin{figure}[H]
    \centering
    \includegraphics[width=1.1\textwidth]{RecIncrease40Key}
    \caption{Caption missing}
    \label{fig:PressConn}
\end{figure}



\section{Mapping}
The mapping of the pumps RPM and flowrate can be seen in \ref{fig:RPM} and \ref{fig:Flowrate}.
\begin{figure}[h]
    \centering
    \includegraphics[width=0.7\textwidth]{RPM.png}
    \caption{RPM of the Pumps at different control signals/duty cycles}
    \label{fig:RPM}
\end{figure}


\begin{figure}[h]
    \centering
    \includegraphics[width=0.7\textwidth]{Flow.png}
    \caption{Flow rate at different control signals/duty cycles}
    \label{fig:Flowrate}
\end{figure}


\section{Design of control algorithms}
In order to design control for driving the system some investigations on System 2, the two pump solution were done. In picture \ref{fig:PreTestReg1}, the results from test with all parameters except the speed of the recycle pump were kept constant, can be seen. \\
\\
In picture \ref{fig:PreTestReg3}, the results from test with all parameters except the speed of the feed pump were kept constant, can be seen.

\begin{figure}[h]
    \centering
    \includegraphics[width=1.65\textwidth, angle = 270]{PreTestReg1.png}
    \caption{Tests with recycle pump as changing parameter}
    \label{fig:PreTestReg1}
\end{figure}

\begin{figure}[h]
    \centering 
    \includegraphics[width=1.65\textwidth, angle=270]{PreTestReg3.png}
    \caption{Tests with feed pump as changing parameter}
    \label{fig:PreTestReg3}
\end{figure}





