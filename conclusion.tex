% !TEX root = /Users/Gela/Desktop/Thesis_latex/thesis.tex

The permeate quality of an RO-system is reduced considerably when the system is running on hot water due to the decreased salt rejection. The results from our tests showed that the most effective way of incresing permeate quality was to create a higher permeate flow when the system was hot to dilute the increased salts in the permeate with a larger volume of water. By identifiying what permeate flows are needed at different temperatures it is possible to find an optimzed permeate flow for every temperature within the operating range..  

By replace the current one pump system without any feedback loops with a system using two pumps and PI regulators and the abilty to control the system allows for a more optimzed performance on both water efficiency and energy consumption. As mentioned, the current system does not use any feedback controllers and therefor it cannot adapt to the changing behavior of the reverse osmosis membrane that is introduced by changing temperatures. The modified system however, can measure and adapt to changing working condition and improve the performance of the membrane.

Since the performance of the membrane improves when it's subjected to more feed side pressure the it can be argued that the system can be optimized by allways applying maximum feed pressure. However, generating pressure costs energy and at low temperatures this extra energy might not be worth the cost even if it improves water efficiency by increasing salt rejection. Consequently, Running the system at low pressure will cost less energy but waste more water due to the decreased salt rejection. Therefore, it is not possible to optimize energy consumption without reducing the water efficiency of the system. As a result, any optimization of the system will be regarding which of these parameters is considered to have the lowest cost.

The model of the reverse osmosis membrane is build around the teoretcal equations presented by DOW. These equations are accurate in describing the flow and pressure charachteristics at different temperature. However, DOW is using a fixed rejection rate for all temperatures and does not incorporate the diluting effect that occurs when pressure is increased.  To be able to make a more accurate model this phenomenon needs to be matematically described and added to the model.

To summarize, the feedback controlled two pump system is a viable soultion with several benefits over the current system and it is possible to optimize water efficiency and power consumtion when using two pumps that would not have been possible with one.  






 

