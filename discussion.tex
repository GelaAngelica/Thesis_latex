% !TEX root = /Users/Gela/Desktop/Thesis_latex/thesis.tex


\section{System behaviour} 



The test setup was built to understand how the performance of the reverse osmosis changed while operating in different conditions and by analysis the logs from the tests it was possible to understand the complexity of this multiple input and output system. Testing was mostly carried out by fixing all parameters but one and then change this parameter and log how the system behaved, this proved to be difficult because some parameters, for instance recirculation flow and feed pressure were directly connected to each other. However, analysis of the log files from the tests it was possible to understand how different conditions and setups affected the behaviour of the system. 

The main purpose of this thesis was to investigate the advantages of using a two-pump system instead of the current one pump system. We found that there were multiple advantages of using the two-pump system, the overall power consumption and noise levels could be reduced. The new setup also had the unforeseen advantage of reducing the pressure drop over the membrane. This was a result of changing the position of the inlet pump from inside of the recirculation loop to pumping straight into the loop and using the recirculation pump to create the desired flow. The reduced pressure drop causes the membrane to be more evenly pressurized, this should in theory prevent the uneven scaling of the membrane and also create a higher permeate flow at the section of the membrane that is closest to the recirculation stream.

Initially it was believed that a higher recirculation flow would improve the performance of the membrane. The initial theory was that by increasing the flow rate over the membrane surface a more turbulent flow would be achieved causing the high salinity water close to the membrane to mix with water further from the membrane. However, tests showed that this was not the case. The performance did improve by a small amount. However, by increasing the recirculation flow the pressure was also increased and the small improvements that could be observed was most likely caused by the increased pressure, not the increased flow rate over the membrane. Therefore, it was concluded that there was a no possibility of effectively optimizing the membrane by increasing the recirculation flow. According to the membrane manufacturer the recovery rate should not exceed 20 \% in order to ensure the longevity of the membrane. Therefore, it was decided to use the recirculation pump to ensure that the recovery remained fixed at 20\%.

Feed pressure affect both the salt rejection of the membrane and also the permeate flow. Increasing feed pressure had a direct relationship to permeate flow at a given temperature. Salt rejection also increased with higher feed pressure. This can be seen in figure TBD and from this data it can also be concluded that the increase is nonlinear. The positive effect of an increased feed pressure is greater at low pressure, for instance an increase from 2 to 4 bar has a larger positive effect than an increase from 7 to 9 bar. The physical reason for the improved salt rejection is due to the reason that water passes through the membrane surface at an increasing rate due to the higher pressure and dissolved salts that pass though the membrane is diluted by the larger flow of water. So by increasing the feed pressure and the permeate flow it is possible to improve salt rejection by diluting the passing salts with a higher permeate flow. 

The higher temperature of the feed water, the lower is the viscosity of the water. Hot water also has a higher diffusion rate than cold water. Thereby, higher temperatures cause higher permeate flow over the membrane and increased salt passage over the membrane. Temperature is a parameter that cannot be controlled and is completely dependent on the temperature of the tap water where the system is located. Tap water temperature may vary from country to country and can also have seasonal changes. The results from tests concluded that temperature was the most significant quantity that needed to be incorporated in order to design an optimal system. 

The conductivity of the feedwater is also a quantity that depends on the tap water and determines the lowest possible conductivity of the recirculation loop. By using permeate conductivity as a setpoint for controlling the drain valve it is possible to optimize the water efficiency of the system. Lower conductivity feed water allows the system to recirculate more water without reaching the critical limit when a permeate conductivity of 30 uS/cm could not be maintained. Therefore, the system could adapt to different feed water conductivity by adjusting the water efficiency.

Net driving pressure depends on the feed side pressure in the recirculation loop, permeate side pressure and also the osmotic pressure caused by the different salt concentrations across the membrane. In order to save energy, it is beneficial to have a low permeate side pressure and lower conductivity in the recirculation loop. In order to save water, the control loop controlling the drain valve accumulate high conductivity water in the recirculation loop which decreases NDP. However, this drawback is necessary to maintain a high water efficiency an in the authors opinion, the slight decrease in NDP is worth the increase in efficiency. It is possible to increase the feed side pressure by increasing the pump speed. However, this requires more energy. The only way to improve the NDP of a system without decreasing water efficiency or increasing energy is to decrease the permeate side pressure. This can be done by selecting components with a low pressure drop and by using a short flow path.

\section{s}imulation

\section{Model based design}


\section{Control System Design}

There three main properties that was to be optimized were water efficiency and energy consumption. The conductivity of the permeate water should also be controlled.

A PI controller was used to control the conductivity of the permeate water by opening or closing the drain valve. Since the salt rejection of the membrane changed due to inlet water temperature and feed pressure there is a point for any salt rejection that generates a permeate conductivity of 30 uS/cm. As can be seen in figure 5.15 from the tests of the current system, a system with room tempered inlet water was able to maintain 30 uS/cm permeate water with 2500 uS/cm in the recirculation loop but if the inlet water was heated to 40 degrees Celsius only 1500 uS/cm recirculation conductivity was needed to reach 30 uS/cm. Thereby, controlling the valve position was critical to being able to make sure that the permeate conductivity was 30 uS/cm regardless of operating conditions. 

Since the recirculation loop and the membrane contains about TBD liters of water the response from changing the drain valve was slow, much longer than the other control loops which has an almost immediate effect. Because of this it was necessary for the controller to slow. 

An increase in pressure result in more fluid getting pushed out of the drain valve without changing the position of the valve. As a result, changes in pressure will act as a disturbance on this control loop. To counter the effect of this the valve needs to close to make sure that no water is unnecessarily being rejected to drain. However, closing the valve will result in an increased pressure in the system that need to be handled by the controller controlling the pressure in the system. This means that these two controllers are coupled to each other and that it is impossible to change one without influencing the other. 
Tests showed that increased recovery did not affect the salt rejection of the system. For this reason, it was decided to set the recovery setpoint to 20 \% which was a the recommended recovery from the membrane manufacturer. This regulator has little impact on the regulator controlling the drain valve but when the flow in the recirculation loop is increased so is the pressure in the loop. The effect of this can be seen in figure 5.23 and 5.22. From the figures it can be seen that 22 percent recovery caused a pressure of 5.2 TBD bars and a recovery of 14 percent TBD increased the pressure to 6.2 bars. Consequently, the regulator pressurizing the recirculation loop act as a disturbance on the recovery regulator.
The last controller was the controller tasked with optimizing the membrane. Warmer inlet temperature increases the amount of salts that pass through the membrane as well as the water. However, the increased temperature increases the salt passage more than the passage of water, resulting in a higher permeate flow with a higher conductivity. The only way to counter this phenomenon and increase salt rejection is to increase the permeate flow by increasing feed pressure and dilute the increased salt in the permeate with more water. By using the formula to convert feed water temperature to a setpoint for the permeate flow the salt rejection of the membrane could be optimized by increasing the feed pump to generate a higher feed pressure and thereby a higher permeate flow. This design allows the system to counter the detrimental effect increase inlet temperature has on the salt rejection of the membrane at the prize of using more energy. The increased salt rejection allows higher conductivity water to be accumulated into the recirculation loop and thereby water efficiency is increased. In figure 5.12 it can be seen that increasing feed pressure has a larger positive effect on salt rejection when the inlet temperature was high and therefor there is no need to waste energy running the system at high pressure because it will have low impact on salt rejection. In this scenario, it is better to save energy by running the system at lower pressure and permeate flow. 
The system was designed to be able to deliver a stream of permeate water with a conductivity of 30 uS/cm without using more energy and rejected water than necessary, but our tests showed that the only way to increase salt rejection and lower the amount of rejected water was to increase the permeate flow. This meant that there is a trade-off between saving water and wasted energy and vice versa. Running the system on low pressure decreases salt rejection and thereby less salts could be recirculated back into the recirculation loop meaning that more water needed to be wasted. By testing, the lowest permeate flow that generated a permeate conductivity of 30 uS/cm at different temperatures could be found and these results were translated into an optimal function for what permeate flow should be used as a setpoint at different temperatures. 

\section{Noise reduction}

An additional positive effect of using two pumps instead of one was that there was a significant reduction in noise. Since this effect was not a part of the initial scope and just a positive side effect of using two pumps no data was gathered to support the claim but the difference could be heard when both systems were tested. The reason for the noise reduction was that both the two pumps were running at much lower rpms when using two pumps than one. 

\section{Reduced pump size}
Using the pumps with the new system the pump speed varied from TBD to TBD depending on the permeate flow setpoint which is a reduction from the 60 \%and 80 \% used when one pump was used. This indicate that it might be possible to use smaller pump. Smaller pumps could allow the system to be smaller and they are often cheaper than larger pumps. This reduction of the prize could reduce the increased cost of buying two pumps instead of one. Smaller pumps might also be quieter.

\section{Membrane size}
membrane size determines the permeate flux and for smaller systems that it might be beneficial to reduce the size of the membrane to make the device smaller and possibly cheaper. In addition to the reduced size and prize a smaller membrane has less pressure drop and more even pressure over the whole membrane.

\section{Membrane identification method}
This report is based on the DOW FilmTec membrane TBD and the temperature to optima permeate flow rate will only work for this membrane. However, the method for finding this curve for any other reverse osmosis membrane is general and easy to follow. By following the method outlined in this report this curve can be found for any other membrane and can be modified for other specifications on the permeate conductivity. 

\section{Scaling}
One advantage of using a permeate flow setpoint for the feed pump controller is that it will let the system adapt to scaling. When scaling occurs, the membrane surface become coated with suspended solids that clogs the surface of the membrane. Using the control design from this thesis, when the membrane surface clogs up more pressure is needed to reach the setpoint for the permeate flow and the feed pump will compensate for this by increasing its speed. Membrane fouling is inevitable and eventually the membrane will have to be replaced. Even though Membrane fouling can't be stopped it can be minimized by following the operational guidelines from the membrane manufacturer. One of these guidelines are the maximum allowed recovery the recovery control loop has been implemented. 

\section{Drain valve}
The valve used in the test rig was built for much larger systems and not intended for fine tuning. This caused problems because it was not possible to make small changes of the position of the valve and this caused oscillations in the conductivity of the permeate. Because it was clear what caused the oscillation and that it would take a lot of time to find, order and rebuilt the test rig with a new valve we choose to continue using the valve. If there are any further development of the system, the valve should be replaced.










